\documentclass[12pt, letterpaper]{article}

\title{My First Latex Document}
\author{Rana Universe\thanks{Mail Us AT: RanaUniverse321@gmail.com}}
\date{August 2025}


\usepackage{lipsum}

\usepackage{graphicx} %LaTeX package to import graphics
\graphicspath{{images/}} %configuring the graphicx package

% \renewcommand{\figurename}{Graph Image}
% To get Figure 1,2,3... i will use normal, if i want custom name i will uncomment this upper, which will say, Graphh Image 1, 2, 3 ...


\usepackage{pgffor} % For loops i will use this package so that i can use repetative work easily



\usepackage{amsmath}

\usepackage{breqn}




\begin{document}

\maketitle






\begin{figure}[htbp]
\centering

\fbox{
\includegraphics[width=0.5\textwidth]{rana_universe_logo_round_black.png}
}

\caption{Rana Universe logo in black circle}

\label{fig:rana-universe-logo}

\end{figure}


\newpage

\noindent\rule{\linewidth}{5pt}

\begin{center}
    \Huge\textbf{Mathematical Indices}
\end{center}

\noindent\rule{\linewidth}{5pt}


Below is the equation representing the sum of squares:

\[
\begin{aligned}
a^2 + b^2 &= c^2 \\
a + b &= c\\
x^{123} &= x + y + z ^ 3\\
y^{m+n} &= y^m \cdot y^n\\
% Below the \text i need to use the \usepackage{amsmath}
x^{1/2} &= \text{Square Root of } x \\
y^{m/n} &= \text{nth root of } y^m \\
b^{\frac{m}{n}} &= \text{This is also nth root of } b^m. \\
b^{\frac{m}{n}} &= \text{This is also nth root of } b^m \text{.} \\
% In This two i should use the upper text mode for full stop.
x^{-3x} &= \text{A incices of variable x with negative} \\
(a^m)^n & = \text{double power}\\
\left(a^m\right)^n &= \text{double power with left right} \\
(\frac{a}{b})^n &= \text{Here i use normal brackets}\\
\left( \frac{a}{b} \right)^n &=\text{Here i sued big brackets} \\
\end{aligned}
\]


\[
f(x) = \left( x + \frac{1}{x + \frac{1}{x + \cdots}} \right) ^ {100}
\]

\[
f(x) = \left( x + \frac{1}{x + 
         \frac{1}{x + 
         \frac{1}{x + 
         \frac{1}{x + 
         \frac{1}{x + 
         \frac{1}{x + 
         \frac{1}{x + 
         \frac{1}{x + 
         \frac{1}{x + 
         \frac{1}{x + 
         \frac{1}{x + 
         \frac{1}{x + 
         \frac{1}{x + 
         \frac{1}{x + 
         \frac{1}{x
         }}}}}}}}}}}}}}}
         \right) ^ {x + x^{-1}}
\]


$$
(\frac5{9})^3 \\
$$

\[
\left(\frac9{987}\right)^{ x^{1/2}}
\]


This below is looks bad, i should not use text inside the math modes.

\[
\begin{aligned}
    \left(a^m\right)^n \\
    \text{This upper is equals to the value of the below:} \\
    a^{mn}
\end{aligned}
\]



\[
\left(a^m\right)^n
\]

\noindent 
This is equal toThis is equal toThis is equal toThis is equal toThis is equal to...
This is equal toThis is equal toThis is equal toThis is equal toThis is equal to...

\[
a^{mn}
\]





\begin{align*}
\left(a^m\right)^n &= a^{mn}  \text{(Power of a power rule)} 
\end{align*}





\[
\begin{aligned}
    \left(a^m\right)^n &= \text{A Indices in Math} \\
    \parbox{0.85\linewidth}{
        \centering 
        \textbf{Still This is bad.} \\
        This is a very large paragraph, this is not good to use in math mode. I should use it at least as little as possible. Using text in math mode is not recommended, but this shows how to use \text{\texttt{\char`\\}\textbf{begin}} environment to show text there.
    } & \\
    a^{mn} &= \text{This is also same} 
\end{aligned}
\]


\[
\begin{aligned}
    x^{y^z}  \\
    \left\{a^{b^c}\right\} \\
    x_{i+1}^{\frac{n}{2}} \\
    & \text{The Eular Equation is below using indices} \\
    e^{i\pi} + 1 = 0 \\
\end{aligned}
\]

$$
F = \frac{G m_1 m_2}{r^2}
$$

\begin{align*}
& F = \frac{G m_1 m_2}{r^2} \\
& (a^m)^n
\end{align*}


\[
\begin{aligned}
    & F = \frac{G m_1 m_2 }{r^2} \\
    & (a^m)^n \\
    \left(a^{m^p}\right)^n \\
    x^{\sqrt{y}} \\
    a^{\frac{m}{n^k}} \\
    a^{\frac{m}{(n^k)}} \\
    a^{\frac{m}{\left(n^k\right)}} \\
    & \text{This below is bad to use as i can thnk} \\
    a^{\displaystyle\frac{m}{n^k}} \\
\end{aligned}
\]





\newpage



This here i need to use to make on new line. This is good to use maybe.


$$
\begin{aligned}
a + b &= c \\
a - b &= d \\
a \times b &= e\\
a + b \\
a - b \\
a \times b \\
a \cdot b \\
a \div b \\
\frac{a}{b} \\
a = b \\
a \neq b \\
\end{aligned}
$$

The below is a separate things a variable like things is this.

$$
\frac{a+b}{c} \times d = x \neq y
$$




Below the equations numbers will be shows per line even no equaions is this.
\begin{align}
    a + b \\
    a - b \\
    a \times b
\end{align}



To Get no numbering i will need align*.
\begin{align*}
    a + b \\
    a - b \\
    a \times b
\end{align*}








\newpage

\begin{center}
    \Huge\textbf{Mathematical Fractions Symbols}
\end{center}

\noindent\rule{\linewidth}{5pt}

\[
\frac{3}{4}
\]

$$
\frac{3}{4}
$$

$\frac{\frac{1}{2}}{\frac{3}{4}}$

Let's i will write a equation here, $\tfrac{1}{2} + \tfrac{2}{3}$ this is good now.


\[
\cfrac{1}{2 + \cfrac{1}{3 + \cfrac{1}{4}}}
\]


\[
\frac{1}{2 + \frac{1}{3 + \frac{1}{4}}}
\]

Let's we will write the physics formula, $\frac{\text{Distance}}{\text{Time}}$

Let's we will write the physics formula, $\cfrac{\text{Distance}}{\text{Time}}$

\[
\frac{x^2 + y^2}{\sqrt{x^2 - y^2}}
\]

\[
\cfrac{x^2 + y^2}{\sqrt{x^2 - y^2}}
\]


\newpage

Inline: $\frac{a}{b}$

Display:
\[
\frac{a}{b}
\]


\[
\tfrac{1}{2}x^2
\]

Let's we will write a good math equation with Normal \textbf{frac}, $9x + \frac{1}{x^2+3x + 2}$

Let's we will write a good math equation with \textbf{tfrac}, $9x + \tfrac{1}{x^2+3x + 2}$

Let's we will write a good math equation with \textbf{dfrac}, $9x + \dfrac{1}{x^2+3x + 2}$

Let's we will write a good math equation with \textbf{cfrac}, $9x + \cfrac{1}{x^2+3x + 2}$




In text: \( f(x) = 1 + \tfrac{1}{x} \)

Display:
\[
f(x) = 1 + \dfrac{1}{x}
\]



Continued using only frac part 1:
\[
f(x) = 1 + \frac{1}{x + \frac{1}{x + \frac{1}{x}}}
\]


Continued using only frac part 2:

This below is method  1
\[
    f(x) = x + \frac{1}{x + \frac{1}{x + \frac{1}{x + \frac{1}{x + \frac{1}{x + \frac{1}{x}}}}}}
    \]
    

This below is method  2, this is good to write easily quickly.

\[
f(x) = x + \frac{1}{x + 
         \frac{1}{x + 
         \frac{1}{x + 
         \frac{1}{x + 
         \frac{1}{x + 
         \frac{1}{x + 
         \frac{1}{x + 
         \frac{1}{x + 
         \frac{1}{x + 
         \frac{1}{x + 
         \frac{1}{x + 
         \frac{1}{x + 
         \frac{1}{x + 
         \frac{1}{x + 
         \frac{1}{x
         }}}}}}}}}}}}}}}
\]





\newpage

A Large fractions using the cfrac.

Below is making in bad ways maybe.
\[
    f(x) = x + \cfrac{1}{x + \cfrac{1}{x + \cfrac{1}{x + \cfrac{1}{x + \cfrac{1}{x + \cfrac{1}{x}}}}}}
\]
    
Below is making in good ways maybe.

(It was write in good manner.)

\[
f(x) = x + \cfrac{1}{x + 
          \cfrac{1}{x + 
          \cfrac{1}{x + 
          \cfrac{1}{x + 
          \cfrac{1}{x + 
          \cfrac{1}{x + 
          \cfrac{1}{x + 
          \cfrac{1}{x + 
          \cfrac{1}{x
          }}}}}}}}}
\]


\noindent\rule{\linewidth}{5pt}



Continued using cfrac:
\[
f(x) = 1 + \cfrac{1}{x + \cfrac{1}{x + \cfrac{1}{x}}}
\]


Continued using dfrac:
\[
f(x) = 1 + \dfrac{1}{x + \dfrac{1}{x + \dfrac{1}{x}}}
\]


Continued using tfrac:
\[
f(x) = 1 + \tfrac{1}{x + \tfrac{1}{x + \tfrac{1}{x}}}
\]















\newpage


The mass-energy equivalence is described by the famous equation

\[E=mc^2\]

discovered in 1905 by Albert Einstein. 
In natural units ($c$ = 1), the formula expresses the identity

\begin{equation}
E=m
\end{equation}


% This is the variable type of things, i will use this below and use the value.
\newcommand{\titlevariable}{Let's Start Using the \textbf{`amsmath'}}

\noindent\rule{\linewidth}{5pt}
\begin{center}
\titlevariable \par
\end{center}
\noindent\rule{\linewidth}{5pt}


\begin{center}
\noindent\rule{\linewidth}{5pt}
% Let's Start Using the \textbf{`amsmath'} \par
\titlevariable \par
\noindent\rule{\linewidth}{5pt}
\end{center}




% This below equaiton will not numbering due to star, and it will not allign as i don't use the &= below
\begin{equation*}
\begin{split}
A = \frac{\pi r^2}{2} \\
  = \frac{1}{2} \pi r^2
\end{split}
\end{equation*}




% The below case it will shows alignment because of &= and good for use a good looking of some equations
\begin{equation} \label{eq1}
\begin{split}
A & = \frac{\pi r^2}{2} \\
 & = \frac{1}{2} \pi r^2
\end{split}
\end{equation}




\begin{equation}
\begin{split}
A & = \frac{\pi r^2}{2} \\
B +C + X     Y  Z& = \frac{1}{2} \pi r^2
\end{split}
\end{equation}


% This below is bad as it exis the page size so all content will not shows here, so i should not use this, to use this normal i need to breake thsi manually.
% \begin{equation}
% \begin{split}
% y & = a + bc + def + xy + 98 + 89x + yt + ty^2 + \alpha + \beta + \gamma + m^2 + n^3 + \frac{a}{b} + \sqrt{x^2 + y^2} + \delta + \epsilon + pq + rs + tu + vw + xyz + 1234
% \end{split}
% \end{equation}



% For this i need to use a math related package called, breqn which help to write long equation goodly without beign cutting by the page size.
\begin{dmath}
y = a + bc + def + xy + 98 + 89x + yt + ty^2 + \alpha + \beta + \gamma + m^2 + n^3 + \frac{a}{b} + \sqrt{x^2 + y^2} + \delta + \epsilon + pq + rs + tu + vw + xyz + \alpha + \beta + \gamma + m^2 + n^3 + \frac{a}{b} + \sqrt{x^2 + y^2} + \delta + \epsilon + pq + rs + tu + vw + xyz + \alpha + \beta + \gamma + m^2 + n^3 + \frac{a}{b} + \sqrt{x^2 + y^2} + \delta + \epsilon + pq + rs + tu + vw + xyz + 1234
\end{dmath}


Let's Write some basic Pythagorean themorem.

$$ a^2 + b^2 = c^2 $$

Here `c' represents the hypotenuse, and `a' and `b' represent the other two sides. 


Below the equation number will start form my own wish like number 1 automatically.

\setcounter{equation}{0}  % Start equation numbering from 0+1

\begin{equation}
    a^2 + b^2 = c^2
    \label{eq:pythagoras1}
\end{equation}


As shown in Equation~\ref{eq:pythagoras1}, the sides of a right triangle follow this relation.


\begin{equation}
    a^2 + b^2 = c^2 
    \tag{Rana 1}
    \label{eq:pythagoras2}
\end{equation}

Upper is a another example of using numbering by myself manually.

As shown in Equation~ \textbf{\ref{eq:pythagoras2}}, which numbering was added manually is now working.



\noindent\rule{\linewidth}{5pt}
\begin{center}
Let's Use some more type of equatoins numbering.
\end{center}
\noindent\rule{\linewidth}{5pt}


\begin{equation}
    a + b = c \tag{000}
\end{equation}


\begin{equation}
    a^3 + b^3 = c^3
    \label{eq:pythagoras3}
\end{equation}


This below 2 manually numbering is not good.
\[
a + b = c \quad \ldots (1)
\]

$$
a + b = c \quad \ldots (1)
$$



\newpage


The well known Pythagorean theorem \(x^2 + y^2 = z^2\) was 
proved to be invalid for other exponents. 
Meaning the next equation has no integer solutions:

\[ x^n + y^n = z^n \]


Here is a famous quote:

\begin{quote}
In physics, the mass-energy equivalence is stated 
by the equation \(E=mc^2\), discovered in 1905 by Albert Einstein.
\end{quote}

And now back to the main text.




\noindent Standard \LaTeX{} practice is to write inline math by enclosing it between \verb|\(...\)|:

\begin{quote}
In physics, the mass-energy equivalence is stated 
by the equation \(E=mc^2\), discovered in 1905 by Albert Einstein.
\end{quote}

\noindent Instead if writing (enclosing) inline math between \verb|\(...\)| you can use \texttt{\$...\$} to achieve the same result:

\begin{quote}
In physics, the mass-energy equivalence is stated 
by the equation $E=mc^2$, discovered in 1905 by Albert Einstein.
\end{quote}

\noindent Or, you can use \verb|\begin{math}...\end{math}|:

\begin{quote}
In physics, the mass-energy equivalence is stated 
by the equation \begin{math}E=mc^2\end{math}, discovered in 1905 by Albert Einstein.
\end{quote}








\newpage

The equation $a + b = c$ is simple.

\[ a^2 + b^2 = c^2\]


\begin{equation}
	a + b = c
\end{equation}

\begin{equation}
a^2 + b^2 = c^2
\end{equation}

\begin{equation}
	a^3 + b^3 = c^3
\end{equation}

\begin{equation}
a^4 + b^4 = c^4
\end{equation}

% This below will start from as usual sentence.
$
a + b,\quad a - b,\quad a \times b,\quad a \div b
$

% This below 2 things are same output, this are just for choice i will use.

\[
a + b,\quad a - b,\quad a \times b,\quad a \div b
\]


\[
a + b,\quad 
a - b,\quad 
a \times b,\quad 
a \div b
\]





\vspace{10\baselineskip}

I love this Upper Examples.



\newpage





\mbox{}


\newpage





\foreach \name in {Rana, Universe, RanaUniverse} {
    Hello, \textbf{\name!} \par
}


\vspace{5em}
\foreach \n in {1,...,9} {
    I am Rana Universe...\textbf{(\n)} \par
}

\vspace{3em}

\foreach \n in {1,...,9} {
    
    \noindent \textbf{\n.} I am Rana Universe... \par

    \textbf{\n.} I am Rana Universe... \par
}



\begin{figure}[ht]   % Start figure environment
    \centering   % Center the image

    % \includegraphics[width=.5\textwidth]{mesh.png}    

    % This below is for adding a black border
    \fbox{
        \includegraphics[width=0.5\textwidth]{mesh.png}
    }

    \caption{\textbf{A nice plot.}}   
    \label{fig:mesh1}   % Label: Internal name used to reference the figure, it should be unique
\end{figure}


As you can see in \textbf{Figure \ref{fig:mesh1}}, the function grows near the origin. This example is on page \pageref{fig:mesh1}.


As you can see in \textbf{\textit{Figure \ref{fig:mesh1}}}, the function grows near the origin. This example is on page \pageref{fig:mesh1}.



\begin{figure}[htbp]
	\centering
	\includegraphics[width=0.5\textwidth]{linux_logo.png}
	\caption{\textbf{Linux Logo}}
	\label{fig:linuxlogo}
\end{figure}

Now in \textit{Figure \ref{fig:linuxlogo}}, you can see the famous Linux logo.
This is shown on page \pageref{fig:linuxlogo}.





\vspace{10em}

I am Rana Universe...

I am Rana Universe...

I am Rana Universe...

I am Rana Universe...

I am Rana Universe...

I am Rana Universe...

I am Rana Universe...

I am Rana Universe...

I am Rana Universe...

\vspace{3em}





\lipsum[1]

\lipsum[2]

\vspace{1em}

\lipsum[1]




% This line here is a comment. It will not be typeset in the document.
\end{document}



